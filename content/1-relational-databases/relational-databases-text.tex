%!TEX root = ../../book.tex

% ******************************* Part: Relational Databases ****************************

%this is a overarching PART that can be replicated to change overarching areas in the book
\part{Relational Databases}
\label{part:relationaldatabases}
This \lcnamecref{part:relationaldatabases} of the book teaches the basics of relational databases. It will walk you through the basics of relational databases and how to design a database from scratch. After this, it will teach you how to create ER and EER diagrams and design and develop a database from an ER model. Finally, it will teach you how to normalize a database to the 4th normal form.

% ******************************* Chapter: Introduction ****************************
\chapter{Introduction}
\label{chap:relational:introduction}
This chapter introduces the fundamental concept of databases. It begins with a clear definition of databases, providing an overview of their purpose and significance in storing and managing information. The chapter then guides the reader through the database creation process, highlighting the critical steps from design to deployment. Essential components of a database, such as tables, fields, keys, and relationships, are explained concisely. This chapter serves as a foundational starting point for those new to databases, offering a clear understanding of their basic structure and components.

\section{What is a database?}
A database is where data can be stored in large quantities in a structured way. It collects organized data to be easily accessed, managed, and updated. A fundamental property of databases is also the fact that they are persistent, meaning that the data is stored on a physical medium, such as a hard drive, and is not lost when the computer is turned off, as well as the fact that it allows you to create dynamic queries, meaning that you can ask complex questions of the data.

A database is a structured and organized collection of data that serves as a centralized repository for storing, managing, and retrieving information. Its primary purpose is to efficiently store and manipulate data to support various applications and processes. At its core, a database consists of two key elements:

\begin{enumerate}
    \item Data: Data is the fundamental building block of a database. It represents the information that needs to be stored and can take various forms, including text, numbers, dates, and multimedia. Data is organized into tables, records, and fields, with each table containing related information and each record representing a distinct data unit. In contrast, fields hold specific attributes or properties of that data.
    \item Database Management System (DBMS): The DBMS is the software that facilitates interactions with the database. It is an intermediary between users or applications and the actual data storage. The DBMS provides essential functionalities such as data retrieval, insertion, updating, and deletion, enforces data integrity and security, and ensures efficient data access through query processing. It also manages concurrency control to support multiple users working with the database without data conflicts.
    \item Querying: is another crucial element of databases. Querying involves retrieving specific information from the database based on predefined criteria or user-defined queries. It allows users to filter, search, and analyze data to extract meaningful insights. Querying is facilitated through a query language (e.g., SQL for relational databases) or query APIs (Application Programming Interfaces) that enable users and applications to interact with the database and request specific subsets of data. Querying is crucial because it empowers users to access and manipulate data flexibly and efficiently, making databases highly versatile for various applications, including research, reporting, decision-making, and data analysis. Whether retrieving a list of products from an inventory database or extracting research findings from a scientific database, querying capabilities are fundamental to harnessing the full potential of a database's stored information.
\end{enumerate}

Regardless of its specific type, a database is vital for efficiently organizing and manipulating data, making it accessible and useful for various purposes, including research, analysis, and application development.

\section{What exactly is a relational database?}
A relational database is a specific type of database management system (DBMS) that organizes and manages data using a structured approach based on the principles of relational algebra. It differs from the broader term "database" in several key ways:

\begin{enumerate}
    \item Data Structure: In a relational database, data is structured into tables, where each table consists of rows (tuples) and columns (attributes, properties). This tabular format is highly organized and allows for representing complex relationships between data entities. Each table represents a distinct entity or concept, and the relationships between these tables are defined through keys, such as primary keys and foreign keys.
    \item Data Integrity: Relational databases enforce data integrity through rules and constraints. These constraints ensure that data remains consistent and accurate. For example, primary keys enforce the uniqueness of each record in a table, while foreign keys establish relationships between tables, maintaining referential integrity. These mechanisms help prevent data anomalies and maintain data quality.
    \item SQL Language: Relational databases use the Structured Query Language (SQL) as the standard interface for querying and manipulating data. SQL provides a robust and standardized way to interact with the database, allowing users to perform operations like querying, inserting, updating, and deleting data. SQL's declarative nature enables users to specify what data they want rather than how to retrieve it.
    \item ACID Properties: Relational databases adhere to the ACID (Atomicity, Consistency, Isolation, Durability) properties to ensure transactional reliability. These properties guarantee that database transactions are processed in a way that maintains data consistency and reliability, even in the presence of system failures.
    \item Schema-Based: Relational databases have a defined schema that outlines the structure of the database, including the tables, their attributes, and the relationships between them. This schema is a blueprint for the data, providing a clear structure that facilitates data organization, consistency, and scalability. Changes to the schema are typically managed with care to maintain data integrity.
\end{enumerate}

The term "database" represents a general data storage and management concept. A relational database is a specific type of database system that follows the principles of data organization, integrity enforcement, and querying through structured tables and SQL. Relational databases are well-suited for applications requiring complex data relationships, consistency, and transactional reliability, making them widely used in various industries and research settings.

\section{Why use a relational database?}
First of all, relational databases tend to be the default choice of database unless other specific requirements or circumstances exist. This default choice is because relational databases are the most mature and widely used database type. This maturity means there is a lot of support for relational databases, and many people know how to work with them. Finding people to work with relational databases is easy, and much documentation and support is available.

\begin{figure}[htbp]
    \centering
    \includegraphics[width=1\textwidth]{content/1-relational-databases/figures/i1-databases-everywhere.png}
    \caption{A system of databases and services}
    \label{fig:0.i1-databases-everywhere.png}
\end{figure}

You interact daily with devices that include databases and services that do the same. Examples are:

\begin{itemize}
    \item Your phone and every single app on your phone
    \item Your computer and most of the applications herein.
    \item All websites you visit, such as your bank, google, etc. (few exceptions exist, but any website that is dynamic and that can save data will most likely also use a database)
    \item Your social media
    \item Your email
    \item Your favorite online store
    \item Your favorite games
\end{itemize}

While databases are everywhere now, this was not always the case. Databases have existed for a long time but were not always as popular as now. The first databases were created in the 1960s and were used for large-scale data processing, while they slowly became more and more used, especially at the end of 1970 and the beginning of 1980. The first relational database was created in 1970 and was called the relational model. It was created by Edgar F. Codd and was based on the mathematical theory of sets and relations. The relational model was a huge success and is still today's most widely used database model. The relational model was so successful that it is often used as a synonym for relational databases.

Initially, relational databases were predominantly utilized in substantial organizations such as banks or for calculating salaries in major corporations. This exclusivity was due to the high cost of the computers required to operate these databases, rendering them unaffordable for the average person. However, this scenario transformed dramatically in the 1980s with the rising popularity and affordability of personal computers. As computers became more accessible, a more significant segment of the population could afford to own and run a database on their personal computers. Consequently, this led to a surge in database usage, significantly increasing the demand for and creation of databases. The growth in demand meant that an ever-increasing number of people began to use databases, further amplifying their prevalence and importance in the digital world.

\section{What is SQL?}
SQL, which stands for Structured Query Language, is the cornerstone of relational databases. It's a specialized programming language designed to manage and manipulate data in a relational database management system (RDBMS). SQL is essential for various operations within these databases, including querying, updating, and managing data.

Relational databases store data in tables akin to spreadsheets with rows and columns. Each row represents a unique record, and each column a specific attribute of the data. The power of relational databases lies in their ability to efficiently organize and retrieve large volumes of data through the use of relations, typically in the form of tables.

SQL plays a pivotal role in this process. It allows users to:

\begin{itemize}
\item \textbf{Query Data:} SQL can retrieve specific data from a database through queries. For instance, if you want to find all customers from a particular city, SQL can quickly filter and display this information.
\item \textbf{Insert and Update Data:} Adding new records or updating existing ones is straightforward with SQL. It ensures data accuracy and integrity while modifying the database.
\item \textbf{Create and Modify Schema:} SQL is used to create the database structure, like tables, and modify it as needed. This includes defining the columns, data types, and constraints.
\item \textbf{Data Manipulation:} Beyond basic queries, SQL can perform complex data manipulations, combining data from multiple tables and executing sophisticated analytical functions.
\end{itemize}

One of SQL's greatest strengths is its widespread adoption and standardization. Most relational database systems use SQL, making it a crucial skill for database professionals. Its syntax and commands are relatively consistent across different database systems, with only minor variations. Understanding SQL is fundamental for anyone looking to work with relational databases, as it opens the door to efficiently managing and utilizing vast sets of data in an organized manner.

\section{What is PostgreSQL?}
PostgreSQL, often simply called Postgres, is an advanced, open-source relational database management system (RDBMS) that stands out in the vast landscape of relational databases. It's renowned for its robustness, scalability, and alignment with SQL standards. Understanding PostgreSQL in the context of relational databases involves appreciating its unique features and the reasons behind its popularity.

\begin{figure}[htbp]
    \centering
    \includegraphics[width=0.4\textwidth]{content/1-relational-databases/figures/PostgreSQL_logo.3colors.540x557.png}
    \caption{PostgreSQL Logo}
    \label{fig:PostgreSQL_logo.3colors.540x557.png}
\end{figure}

Choosing PostgreSQL for database management comes with several advantages:

\begin{itemize}
    \item \textbf{Open Source:} Being open-source, it's free to use, modify, and distribute. This aspect makes it particularly attractive for startups and companies looking to reduce costs without sacrificing quality.
    \item \textbf{Community-Driven Development:} PostgreSQL benefits from a vibrant community that continuously contributes to its development, ensuring the database is always evolving to meet user needs.
    \item \textbf{Reliability and Stability:} It's known for its data integrity and resilience. Businesses can rely on PostgreSQL for critical applications requiring consistent uptime and robustness.
    \item \textbf{Flexibility for Developers:} PostgreSQL's support for various programming languages and its extensibility make it highly adaptable for various applications.
    \item \textbf{Scalability:} It handles large volumes of data effectively, making it suitable for businesses anticipating growth in data volume and user load.
\end{itemize}

PostgreSQL represents a sophisticated and reliable choice within the realm of relational databases. Its adherence to SQL standards and its open-source nature make it a formidable tool for businesses and developers seeking a robust, scalable, and cost-effective database solution. The most used relational database management systems are Oracle, MySQL (Hereunder MariaDB and other variants), Microsoft SQL Server, PostgreSQL, IBM Db2, and SQLite.

This book currently uses PostgreSQL but will add examples in MySQL (or variants), MSSQL, and SQLite.


\section{Getting to terms with the terminology}
Understanding the realm of database systems requires familiarity with several core terminologies. Let's delve into each of these terms and explore their interrelationships.

\subsection{Overall Database Terms}
The terms one needs to understand, and how they relate to each other, are shown in \cref{fig:1.dbms-definitions.png}.


\begin{enumerate}
    \item \textbf{Database Systems:} A Database System is an integrated set of software tools that allows users to store, modify, and extract information from a database. It encompasses the DBMS software, the database itself (which includes the schema definitions and stored data), and the user application programs or queries.
    \item \textbf{User Application Programs or Queries:} User Application Programs or Queries refer to the software and commands that interact with the database system. These can range from simple query commands in SQL to complex programs written in programming languages like Python, Java, or C\#, designed to manipulate or retrieve data from the database.
    \item \textbf{DBMS Software:} DBMS Software, or Database Management System Software, is the core component of a database system. It acts as an intermediary between the user and the database. The DBMS manages the stored data, ensuring its integrity, security, and consistency. It also handles tasks such as data retrieval, update, and administration.
    \item \textbf{Schema Definitions:} Schema Definitions in a database system represent the logical structure of the entire database. They define how data is organized and how the relationships among different data elements are established. The schema includes definitions for tables, columns, data types, constraints, and relationships. It's like a blueprint for the database, dictating its organization and how the data within it is related.
    \item \textbf{Stored Data:} Stored Data is the actual data that resides in the database. This is the collection of information that has been stored in accordance with the schema definitions. Stored data can include various types of data, such as textual data, numerical data, dates, or binary data, depending on the nature of the database and its schema.
\end{enumerate}

\begin{figure}[htbp]
    \centering
    \includegraphics[width=1\textwidth]{content/1-relational-databases/figures/1.dbms-definitions.png}
    \caption{High Level Explanation of a Database Management System}
    \label{fig:1.dbms-definitions.png}
\end{figure}

The relationship among these components is hierarchical and interdependent. At the base, we have the \textbf{Stored Data}, which is the essence of the database. This data is structured and organized according to the \textbf{Schema Definitions}. The \textbf{DBMS Software} serves as the mediator between the stored data and the end-users. It uses the schema definitions to ensure data is correctly stored and retrieved. \textbf{User Application Programs or Queries} interact with the DBMS software to perform operations on the data — whether it's querying for specific information, updating records, or performing analyses. All these components together constitute the \textbf{Database System}, which is the overall environment enabling data management and utilization.


\begin{figure}[htbp]
    \centering
    \includegraphics[width=1\textwidth]{content/1-relational-databases/figures/1.DatabaseAndTerminologyMap.drawio.png}
    \caption{Terminology extended as databases became more complex}
    \label{fig:1.dbms-content-anatomy.png}
\end{figure}

As databases became more complex, the terminology also became more complex. The terms one needs to understand and how they relate to each other are shown in \cref{fig:1.dbms-content-anatomy.png}. A computer also called a server if its only purpose is to host services for different machines, can have multiple installed DBMSs of a single type. Each installation is separate, and has its own users, and listens to its own port for connections. These separately installed DBMS' are called instances and are typically created when a company or customer needs full rights to the DBMS system to create new users and more. Each instance of the DBMS is able to hold multiple databases. Databases can have its own users, and are usually the first tool to reach for when separating access to data, where creating new instances should only be done if separate databases does not satisfy the requirements of the system implementor. Each of theses databases, can in turn, hold multiple tables. Each table can contain multiple rows, and each row can contain multiple columns. Each column can contain multiple cells, and each cell can contain a single value.

\begin{figure}[htbp]
    \centering
    \includegraphics[width=1\textwidth]{content/1-relational-databases/figures/1.DatabasesAndConnecting.drawio.png}
    \caption{How do we connect to the database?}
    \label{fig:1.dbms-connection-explained.png}
\end{figure}

To connect to the database we want, an application would need to either connect to an instance on the localhost or an instance on another server. In \cref{fig:1.dbms-connection-explained.png}, we explore how this works. The green arrow would be connecting to a DBMS server within the same physical machine, also often called localhost. It tends to be called localhost, as localhost will always resolve to 127.0.0.1, which is the loopback that is always the computer you are on. The application will need to create a connection string, and this string contains the IP (in this case localhost), the user, the password, and the name of the database the application wishes to connect to. When following the yellow arrow though, we are trying to connect to a different computer to get our data. For this, we need the same information as before, but the application now needs the IP from the remote server in the connection string. Once that detail is changed, the connection can now commence, but the connection will be slower as the data needs to traverse the network. An application will need multiple connection strings and connections over the network if it wants data from various databases, or from multiple instances.

\subsection{The anatomy of a table}
As can be seen in \cref{fig:2.table-definitions.png}, a table is structured through a series of rows and columns that intersect to form cells. Each column in the table is headed by an attribute, which is also commonly referred to as a field, column, or property. These attributes represent the data categories within the table, defining the kind of data each column holds. Similarly, each row in the table is known as a tuple, which can alternatively be called a row, record or an entry. Tuples are instances of the data, encompassing a unique set of values for the attributes defined by the columns. Together, attributes and tuples constitute the fundamental framework of a table.

\begin{figure}[htbp]
    \centering
    \includegraphics[width=1\textwidth]{content/1-relational-databases/figures/2.table-definitions.png}
    \caption{Table Terminology (Artist: Chris Martin)}
    \label{fig:2.table-definitions.png}
\end{figure}

\section{Setting up your machine}
Before you can start working with databases, you need to set up your machine. This section will guide you through the process of setting up your machine, and installing the required software.

\subsection{Windows Specific Instructions}
Go to \url{https://www.enterprisedb.com/downloads/postgres-postgresql-downloads} and download the latest version of PostgreSQL. 
\begin{figure}[htb]
    \centering
    \includegraphics[width=1\textwidth]{content/1-relational-databases/figures/1.install-for-windows-1.png}
    \caption{Download location for PostgreSQL}
    \label{fig:1.postgresql-download-1.png}
\end{figure}

Run the installer, and follow the instructions, but be sure to deselect "Stack Builder" (see \cref{fig:1.postgresql-download-2.png}), as we will not be using it in this book. 

\begin{figure}[htb]
    \centering
    \includegraphics[width=0.6\textwidth]{content/1-relational-databases/figures/1.install-for-windows-2.png}
    \caption{Components to install}
    \label{fig:1.postgresql-download-2.png}
\end{figure}

When asked for a password, use the password you want to use for the postgres user. This is the user that is used to manage the database server. Be sure to remember this password, as you will need it later. After this password it set, the database server instance will have the user postgres, and the password will be what you set. Beware that your computers password, the database server instance password, and the pgAdmin password are all different passwords.

\begin{figure}[htb]
    \centering
    \includegraphics[width=0.7\textwidth]{content/1-relational-databases/figures/1.install-for-windows-3.png}
    \caption{Pick a password for the postgres user}
    \label{fig:1.postgresql-download-3.png}
\end{figure}


This is not the password you will use to connect to the database server but the password you will use to manage the database server. This password is not used in this book, but it is good practice to set it to something you can remember.

\subsection{Mac Specific Instructions}
Installation of PostgreSQL on a Mac is done through the terminal. You can pick any way to do this, but we are going to use the homebrew (https://brew.sh) solution. Open the terminal, and run the following command to install homebrew: 

\begin{minted}[fontsize=\footnotesize]{bash}
    # Install homebrew
    # path broken into two lines for readability
    # if you copy this, make sure to remove the line break
    /bin/bash -c "$(curl -fsSL https://raw.githubusercontent.com
        /Homebrew/install/HEAD/install.sh)"
\end{minted}

Or copy it directly from the frontpage of the homebrew website. After homebrew is installed, you can install PostgreSQL by running the following command in the terminal:

\begin{minted}[fontsize=\footnotesize]{bash}
    brew install postgresql
\end{minted}

After PostgreSQL is installed, you can start the server by running the following command in the terminal:

\begin{minted}[fontsize=\footnotesize]{bash}
    brew services start postgresql
\end{minted}

This will start the PostgreSQL server, and it will start automatically every time you start your computer. If you want to stop the server, you can run the following command in the terminal:

\begin{minted}[fontsize=\footnotesize]{bash}
    brew services stop postgresql
\end{minted}

And if you only want to start the server once, you can run the following command in the terminal:

\begin{minted}[fontsize=\footnotesize]{bash}
    brew services run postgresql
\end{minted}


PostgreSQL will not have a password set at this point, and a user needs to be created, and a password needs to be set. It can be set by running the following command in the terminal:

\begin{minted}[fontsize=\footnotesize]{bash}
    # Create the postgres user
    createuser -s postgres
    # Sometimes the createuser script is located in /usr/local/bin 
    # or /opt/homebrew/bin/ instead, which requires running the 
    # following command instead (substitute the path with the 
    # correct path)
    /usr/local/bin/createuser -s postgres

    # Set the password for the postgres user, at the second
    # command, you will be prompted to write a password
    psql postgres
    \password postgres
    \q
\end{minted}

Running the above code will prompt you to write a password for the postgres user. Remember this password, as you will need it every time you connect to the database server instance. 


Remember to also install pgAdmin using homebrew, by running the following command in the terminal:
\begin{minted}[fontsize=\footnotesize]{bash}
    brew install pgadmin4
\end{minted}

At the first run of pgAdmin a password needs to be set. Remember it, as you need it every time you used pgAdmin. Be aware that this password is not the same as the password for the database server instance, and the password for the postgres user.


\subsection{Linux Specific Instructions}
With Linux, you are mostly on your own, but generally, most Linux distributions have a package manager that you can use to install PostgreSQL. The package manager is different from distribution to distribution, but the most common ones are apt, yum, and dnf. To install PostgreSQL on Ubuntu, you can run the following command in the terminal:

\begin{minted}[fontsize=\footnotesize]{bash}
    sudo apt-get install postgresql
\end{minted}

After PostgreSQL is installed, you can start the server by running the following command in the terminal:

\begin{minted}[fontsize=\footnotesize]{bash}
    sudo systemctl start postgresql
\end{minted}

This command will start the PostgreSQL server and automatically start every time you start your computer. If you want to stop the server, you can run the following command in the terminal:

\begin{minted}[fontsize=\footnotesize]{bash}
    sudo systemctl stop postgresql
\end{minted}

PostgreSQL will not have a password set at this point, and a password needs to be set using the comandline. It can be set by running the following command in the terminal:

\begin{minted}[fontsize=\footnotesize]{bash}
    # Create the postgres user
    createuser -s postgres

    # Set the password for the postgres user
    psql postgres
    \password postgres
    \q
\end{minted}

Running the above code will prompt you to write a password for the postgres user. Remember this password, as you will need it every time you connect to the database server instance. 
Remember to install pgAdmin too. This can be done using the package manager of your distribution. With Ubuntu, you can run the following command in the terminal:

\begin{minted}[fontsize=\footnotesize]{bash}
    sudo apt-get install pgadmin4
\end{minted}

At the first run of pgAdmin, a password needs to be set. Remember it, as you need it every time you use pgAdmin. Be aware that this password is not the same as the password for the database server instance, and the password for the postgres user.

\section{Connecting to the database with pgAdmin}
When you have installed PostgreSQL, you can connect to the database using pgAdmin. pgAdmin is a graphical user interface that allows you to interact with the database. It is a powerful tool that allows you to create, manage, and query databases. This section will guide you through the process of connecting to the database using pgAdmin.

First start the application in your respective operating system. The application will look like \cref{fig:1.pgadmin1}.
\begin{figure}[H]
    \centering
    \includegraphics[width=0.7\textwidth]{content/1-relational-databases/figures/pgadmin/1.png}
    \caption{Starting pgAdmin}
    \label{fig:1.pgadmin1}
\end{figure}

Next, you need to register a server. This is done by right clicking on the "Servers" node in the tree view, and selecting "Create" and then "Server...". This will open a dialog where you can fill in the connection information. The dialog will look like \cref{fig:1.pgadmin2}.

\begin{figure}[H]
    \centering
    \includegraphics[width=0.3\textwidth]{content/1-relational-databases/figures/pgadmin/2.png}
    \caption{Initiate registration of a server}
    \label{fig:1.pgadmin2}
\end{figure}


% The dialog will look like \cref{fig:1.pgadmin3}.

% \begin{figure}[H]
%     \centering
%     \includegraphics[width=0.6\textwidth]{content/1-relational-databases/figures/pgadmin/3.png}
%     \caption{Pick a name, i recommend localhost}
%     \label{fig:1.pgadmin3}
% \end{figure}

Pick a name for the server, and click save. I recommend localhost, as it is the default name, and as it refers to your local computer. This reference is a convention for development machines. 
Move to the tab "Connection" and fill in the connection information. The hostname is localhost, the username is postgres, and the password is the password you set for the postgres user. The dialog will look like \cref{fig:1.pgadmin4}.

\begin{figure}[H]
    \centering
    \includegraphics[width=0.6\textwidth]{content/1-relational-databases/figures/pgadmin/4.png}
    \caption{Fill in connection information, hostname is localhost, username is postgres, and the password is the password you set for the postgres user}
    \label{fig:1.pgadmin4}
\end{figure}

% If you have filled in the information correctly, you will see a success message like \cref{fig:1.pgadmin5}. If you see this message, you have successfully connected to the database server instance.

% \begin{figure}[h]
%     \centering
%     \includegraphics[width=0.7\textwidth]{content/1-relational-databases/figures/pgadmin/5.png}
%     \caption{Success looks like this!}
%     \label{fig:1.pgadmin5}
% \end{figure}
You should now be connected to the database and see it in the right pane.
Now you can create a new database. This creation is achieved by right-clicking on the "Databases" node in the tree view and selecting "Create" and then "Database...". This action will open a dialog where you can fill in the database information. The dialog will look like \cref{fig:1.pgadmin6}.

\begin{figure}[H]
    \centering
    \includegraphics[width=0.5\textwidth]{content/1-relational-databases/figures/pgadmin/6.png}
    \caption{Creating a new database from pgAdmin}
    \label{fig:1.pgadmin6}
\end{figure}

Pick a name for the database, and click save. The dialog will look like \cref{fig:1.pgadmin7}.

\begin{figure}[H]
    \centering
    \includegraphics[width=0.7\textwidth]{content/1-relational-databases/figures/pgadmin/7.png}
    \caption{Pick a database name, and click save}
    \label{fig:1.pgadmin7}
\end{figure}

Now you have created a database, and you can start working with it. You can begin by creating tables and inserting data into them. This creation is achieved by right-clicking on the database and selecting "Query Tool." This will open a dialog where you can write your code, and execute it. The dialog will look like \cref{fig:1.pgadmin8}.

\begin{figure}[H]
    \centering
    \includegraphics[width=0.4\textwidth]{content/1-relational-databases/figures/pgadmin/8.png}
    \caption{Mark the database, and press the query tool}
    \label{fig:1.pgadmin8}
\end{figure}

Write your code and execute it. The dialog will look like \cref{fig:1.pgadmin9}. You are now ready for coding.

\begin{figure}[H]
    \centering
    \includegraphics[width=0.7\textwidth]{content/1-relational-databases/figures/pgadmin/9.png}
    \caption{Write your code, and execute it}
    \label{fig:1.pgadmin9}
\end{figure}

Now that your machine is connected to the database, you can start working with it. The next chapter will guide you through creating a database, managing tables, and querying the database.


% ******************************* Chapter: Relational Database Basics ****************************
\chapter{Relational Database Basics}
\label{chap:relational:relational-database-basics}
This chapter contains the basic building blocks for basic relational databases.
It will explain the basic concepts of databases and how to create, manage, and query a database.

\section{Creating your first database}
Creating a database is the first step in working with databases. This section will guide you through the process of creating a database. 

The following code snippet shows the available syntax for the create database operation. The syntax is the same for all database management systems, but options might differ. The options available in PostgreSQL are shown in the example.

\begin{minted}[fontsize=\footnotesize]{postgresql}
    -- Creating a database with full syntax
    CREATE DATABASE database_name
    WITH
        [OWNER =  role_name]
        [TEMPLATE = template]
        [ENCODING = encoding]
        [LC_COLLATE = collate]
        [LC_CTYPE = ctype]
        [TABLESPACE = tablespace_name]
        [ALLOW_CONNECTIONS = true | false]
        [CONNECTION LIMIT = max_concurrent_connection]
        [IS_TEMPLATE = true | false ]
\end{minted}

The key element to understand here is that the name of the CREATE DATABASE statement and a database name of your choosing are mandatory; the WITH clause and everything after it are optional. The database will be created with default settings if no parameters are set. Therefore, the simplest way to create a database is to run the following command:

\begin{minted}[fontsize=\footnotesize]{postgresql}
    -- Creating an ACME database with minimal syntax
    CREATE DATABASE Acme;
\end{minted}

Note that both examples you have seen until now also use comments. A double dash "--" creates a one-line comment, and a slash-star "/*" creates a multi-line comment. Comments are not executed and are only there to help you understand the code. However, it makes the code more readable and understandable, and it is an excellent practice to use comments.

If you wish to create a database with specific settings, you can use the WITH clause. The following example shows how to create a database with specific settings:

\begin{minted}[fontsize=\footnotesize]{postgresql}
    -- Creating a ACME database with enabled options
    CREATE DATABASE Acme 
    WITH
        OWNER = postgres
        CONNECTION LIMIT = 50;
        IS_TEMPLATE = false;
\end{minted}

\subsection{Altering and deleting databases}
If you wish, an already existing database's settings can be altered. The following example shows how to alter a database and turn off index scans (which is not recommended):

\begin{minted}[fontsize=\footnotesize]{postgresql}
    -- Alter Database Acme to disable index scans
    ALTER DATABASE Acme SET enable_indexscan TO off;
\end{minted}

The last missing component is how to delete a database. The following example shows how to delete a database. Beware that this is a permanent operation and that all data in the database will be lost, and there is no way to recover the data!

\begin{minted}[fontsize=\footnotesize]{postgresql}
    -- Delete a table
    DROP DATABASE Acme;
\end{minted}

\subsection{Switch databases}
Switching between databases within the same script on most database management systems is also possible. Inside the script, one can freely change databases and run queries on different databases. The following example shows how to switch databases in Microsoft SQL:

\begin{minted}[fontsize=\footnotesize]{sql}
    /* Change to database_name
     * This does not work with PostgreSQL */
    USE database_name;
\end{minted}

However, this does not work with PostgreSQL. PostgreSQL is implemented differently and requires that you actively, with your graphical user interface or with the command line, switch to the database you want to work with. The following example shows how to switch databases in PostgreSQL from the command line using psql:

\begin{minted}[fontsize=\footnotesize]{psql}
    -- Change to database_name from psql command line
    \connect database_name
\end{minted}

\section{Datatypes in Databases}
When creating a table, you need to specify the data type for each column. Therefore, this section clarifies what datatypes are, how they should be used, and some best practices. The data type specifies what type of data the column can hold. While databases support different datatypes, several common types work between DBMS implementations. Table \ref{tab:1.postgresql-datatypes} shows PostgreSQL's most commonly used data types.

\begin{table}[htb]
    \centering
    \resizebox{\textwidth}{!}{%
    \begin{tabular}{|l|l|}
    \hline
    \textbf{Data Type} & \textbf{Description} \\ \hline
    \textbf{BOOLEAN} & Logical Boolean (true/false) \\ \hline
    \textbf{CHAR(n)} & Fixed-length character string \\ \hline
    \textbf{VARCHAR(n)} & Variable-length character string \\ \hline
    \textbf{TEXT} & Variable-length character string \\ \hline
    \textbf{INTEGER} & Signed integer (4 bytes) \\ \hline
    \textbf{BIGINT} & Signed integer (8 bytes) \\ \hline
    \textbf{DECIMAL(precision, scale)} & Exact numeric of selectable precision \\ \hline
    \textbf{NUMERIC(precision, scale)} & Exact numeric of selectable precision \\ \hline
    \textbf{REAL} & Single precision floating-point number (4 bytes) \\ \hline
    \textbf{DOUBLE PRECISION} & Double precision floating-point number (8 bytes) \\ \hline
    \textbf{DATE} & Calendar date (year, month, day) \\ \hline
    \textbf{TIME} & Time of day (hour, minute, second) \\ \hline
    \textbf{TIMESTAMP} & Date and time (no time zone) \\ \hline
    \textbf{TIMESTAMPTZ} & Date and time (with time zone) \\ \hline
    \textbf{INTERVAL} & Time interval \\ \hline
    \textbf{UUID} & Universally unique identifier \\ \hline
    \textbf{XML} & XML data \\ \hline
    \textbf{JSON} & JSON data \\ \hline
    \textbf{ARRAY} & Array of any data type \\ \hline
    \end{tabular}%
    }
    \caption{Commonly used data types in PostgreSQL}
    \label{tab:1.postgresql-datatypes}
\end{table}

The datatypes are just some of the available types, and I recommend you read up on the documentation for the database management system you are using. The documentation will show you all the available data types and how to use them.
You should get familiar with a few data types as they will be used most often, and you must understand how they are used and how they differ. Examples of these datatypes are mainly within three areas: text, numbers, and dates.

You have the CHAR(n), VARCHAR(n), and TEXT datatypes within the text area. The CHAR(n) datatype is a fixed-length character string, and the VARCHAR(n) is a variable-length character string. The TEXT datatype is a variable-length character string used for large amounts of text. The difference between CHAR(n) and VARCHAR(n) is that CHAR(n) will always be n characters long and will be padded with spaces if the text is shorter than n characters. VARCHAR(n) will only be as long as the text and will not be padded with spaces. The TEXT datatype is used for large amounts of text and is the most flexible of the three. However, when learning databases, I recommend using the VARCHAR(n) datatype, which is flexible and supports most scenarios. So, for now, think of text as VARCHAR(n). Once you understand databases better, you will understand when to use the other datatypes.

The next datatype area to get familiar with is the numbers area. The INTEGER and BIGINT datatypes are used for whole numbers, and the DECIMAL(precision, scale) and NUMERIC(precision, scale) are used for decimal numbers. The difference between INTEGER and BIGINT is the size of the number they can hold. INTEGER can have numbers from -2147483648 to 2147483647, and BIGINT can contain numbers from -9223372036854775808 to 9223372036854775807. The DECIMAL(precision, scale) and NUMERIC(precision, scale) datatypes are used for decimal numbers, and the difference between them is that DECIMAL is the same as NUMERIC. Still, the implementation can differ between database management systems. The precision is the total number of digits, and the scale is the number of digits to the right of the decimal point. So DECIMAL(5,2) can hold numbers from -999.99 to 999.99. I recommend using INTEGER for whole numbers and DECIMAL(precision, scale) for decimal numbers and leaving other options for when those do not satisfy the requirements.

Lastly, the dates area is the last area to get familiar with. The DATE datatype is used for calendar dates (does not contain the concept of hours, minutes, and seconds), and the TIME datatype is used for the time of day (does not contain the idea of a date). The TIMESTAMP datatype is used for date and time, so it combines the two previous concepts, and the TIMESTAMPTZ datatype is used for date and time with time zone. The INTERVAL datatype is used to quantify the time between two dates or timestamps. I recommend using the TIMESTAMP datatype for date and time and the DATE datatype for calendar dates. The other data types are used for specific scenarios and should be used when those scenarios are present.

\section{Constraints}
Another critical concept to understand when working with databases is constraints. Without constraints, it is impossible to make large-scale, consistent databases. Constraints are used to enforce rules on the data in the database. This section will guide you through the types of constraints that will be used in later sections. The available types of constraints are shown in \cref{tab:1.postgresql-constraints}. We will touch upon four of the above constraint types: primary keys, foreign keys, unique keys, and not null keys. With Primary Keys, we are also going to touch upon the concept of composite keys.

\begin{table}[htb]
    \centering
    \resizebox{\textwidth}{!}{%
    \begin{tabular}{|l|l|}
    \hline
    \textbf{Constraint} & \textbf{Description} \\ \hline
    \textbf{NOT NULL} & Ensures that a column cannot have a NULL value \\ \hline
    \textbf{UNIQUE} & Ensures that all values in a column are different \\ \hline
    \textbf{PRIMARY KEY} & \makecell[l]{A combination of a NOT NULL and UNIQUE. \\Uniquely identifies each row in a table} \\ \hline
    \textbf{FOREIGN KEY} & Uniquely identifies a row/record in another table \\ \hline
    \textbf{CHECK} & Ensures that all values in a column satisfies a specific condition \\ \hline
    \textbf{EXCLUSION} & \makecell[l]{Ensures that if any two rows are compared on the \\ specified column or expression using the specified operator, \\at least one of the comparisons will return FALSE or NULL} \\ \hline
    \end{tabular}%
    }
    \caption{Commonly used constraints in PostgreSQL}
    \label{tab:1.postgresql-constraints}
\end{table}

\begin{itemize}
    \item \textbf{Primary Key:} A primary key is a field in a table that uniquely identifies each row/record. It is a unique identifier for the rows in the table. A primary key cannot contain NULL values, and only one primary key can be in a table. A primary key can be made up of one or more columns. When a primary key is made up of more than one column, it is called a composite key. Make sure your table \textbf{always} has a primary key, as it is the most important constraint in a table. With a primary key, you will often see a datatype of SERIAL, a special datatype used to auto-increment the column's value. Its underlying datatype is INTEGER. This detail is important when making foreign keys to the primary key, as the foreign key's data type must match the primary key's datatype.
    \item \textbf{Foreign Key:} A foreign key is a field in a table that is a primary key in another table. It can be used to link two tables together. Unless otherwise specified, a foreign key can contain NULL values and multiple foreign keys in a table. A foreign key ensures that the value in the foreign key column exists in the primary key column in the other table, which is used to enforce referential integrity and link two tables together.
    \item \textbf{Unique:} A unique constraint ensures that all values in a column are different. It is used to enforce the uniqueness of the values in a column. It is commonly used in conjunction with fields containing an email address or a username to ensure that the user does not exist multiple times, but of course, it finds use in other scenarios as well.
    \item \textbf{Not Null:} A not-null constraint ensures a column cannot have a NULL value. What is a null value? A null value is a value that is not known, not available or does not exist. It is different from a zero value or a field that contains spaces. A null value is used to represent a missing value and is used to describe the absence of a value.
\end{itemize}

Now that you have a basic understanding of constraints, you can start creating tables. The following section will guide you through creating a table and how to use the constraints you have learned about.

\section{Working with Tables}
Now that you have a basic understanding of databases, how to create a database, and the data types and constraints you will encounter in this section, you are ready to start creating tables. This section will guide you through creating a table using the constraints you have learned about.

\subsection{Creating tables}
The following code snippet shows the available syntax for the create table operation. The syntax is the same for all database management systems, but options might differ. The options available in PostgreSQL are shown in the example.

\begin{minted}[fontsize=\footnotesize]{postgresql}
    -- Create a table
    CREATE TABLE [IF NOT EXISTS] table_name (
        column1 datatype(length) column_constraint,
        column2 datatype(length) column_constraint,
        ...
        table_constraints
    );
\end{minted}

What is essential to understand here is that the name of the CREATE TABLE statement and a table name of your choosing are mandatory; the IF NOT EXISTS clause is optional. Each line under the CREATE statement defines a column in the table. The column name, the datatype, and the column constraint is mandatory, and the length is optional. The table constraints are also optional and are used to define constraints on the table. Moving this example to a concrete table implementation, the example below shows how to create a table with the name account and the columns user\_id, username, and password.

\begin{minted}[fontsize=\footnotesize]{postgresql}
    -- Create a table
    CREATE TABLE tableName (
        tableName_id SERIAL PRIMARY KEY, -- auto incrementing id
        username VARCHAR (50) UNIQUE NOT NULL, -- unique username
        password VARCHAR (250) NOT NULL -- never store in plain text
    );
\end{minted}

Specifically, from the CREATE statement, we define a table name here. The following line represents a PRIMARY KEY to ensure the line is unique and can be referred to. This line employs the SERIAL datatype to auto-increment the column's value with each new row added to the table. The line below defines a username with a maximum length of 50 characters but specifies that it must be NOT NULL and that no other row in the table can have the same value. The last line represents a password that can be the maximum length of 250 characters but determines it must be set as the NOT NULL constraint is set. For example, never store passwords in plain text, and always use a hashing algorithm to hash the password before storing it in the database.

The following example shows how to create a table with a foreign key. The example shows how to create a table with the name blog\_entries and the columns id, header, body, and created\_by. The created\_by column is a foreign key referencing the account table's id column. 

\begin{minted}[fontsize=\footnotesize]{postgresql}
    -- Create account table
    CREATE TABLE account (
        id serial PRIMARY KEY,
        username VARCHAR (50) UNIQUE NOT NULL,
        created_on TIMESTAMP NOT NULL, 
        last_login TIMESTAMP
    );
    
    /* Create blog entries table
     * the created_by column references the 
     * id column in the account table */
    CREATE TABLE blog_entries (
        id serial PRIMARY KEY, 
        header VARCHAR (255) NOT NULL,
        body TEXT NOT NULL,
        created_by INTEGER NOT NULL REFERENCES account (id)
    );
\end{minted}

Note how a foreign key is defined and how it REFERENCES the id column in the account table and does not use its definition as a keyword. So, REFERENCES table(id) is the actual implementation of a FOREIGN KEY. This key ensures a blog post cannot be created without a user to be associated with it. This constraint also means that if a user is deleted and cascading deletions are enabled, all blog posts related to that user will also be deleted. Usually, this is not the case, and a deletion will fail until all blog entries from that account are deleted.

Now that we can create tables, the question arises: what happens if something is done wrong, and the table needs to be altered or deleted? 

\subsection{Altering tables}
If you wish to alter a table, you can use the ALTER TABLE statement. The following code snippet shows the available syntax for the ALTER TABLE operation. The syntax is the same for all database management systems, but options might differ. The options available in PostgreSQL are shown in the example.

\begin{minted}[fontsize=\footnotesize]{postgresql}
    -- Alter a table
    ALTER TABLE table_name action;
\end{minted}

As you can see, the alter command does not specify what action to take. The action is set in the following line, and the available actions are shown in \cref{tab:1.postgresql-alter-table-actions}.

\begin{table}[htb]
    \centering
    \resizebox{\textwidth}{!}{%
    \begin{tabular}{|l|l|}
    \hline
    \textbf{Action} & \textbf{Description} \\ \hline
    \textbf{ADD} & Adds a new column to the table \\ \hline
    \textbf{DROP} & Removes a column from the table \\ \hline
    \textbf{ALTER} & Modifies the data type of a column in the table \\ \hline
    \textbf{RENAME} & Renames a column in the table \\ \hline
    \textbf{RENAME TO} & Renames the table \\ \hline
    \textbf{SET} & Changes the value of a column in the table \\ \hline
    \textbf{RESET} & Resets the value of a column in the table \\ \hline
    \textbf{OWNER TO} & Changes the owner of the table \\ \hline
    \end{tabular}%
    }
    \caption{Available actions for the ALTER TABLE statement}
    \label{tab:1.postgresql-alter-table-actions}
\end{table}

From this point on, the task comes to finding the correct action and syntax to use. The following example shows how to add a column to a table.

\begin{minted}[fontsize=\footnotesize]{postgresql}
    -- Add a column to a table
    ALTER TABLE table_name 
    ADD COLUMN column_name datatype column_constraint;
\end{minted}

For specific examples of how to use ALTER for each action, I recommend visiting \url{https://www.postgresqltutorial.com/postgresql-tutorial/postgresql-alter-table/} and reading up on the documentation for the database management system you are using. The documentation will show you all the available actions and how to use them.


\subsection{Deleting tables}
Compared to altering tables, deleting tables is a more straightforward operation. The following code snippet shows the available syntax for the delete table operation. The syntax is the same for all database management systems, but options might differ. The options available in PostgreSQL are shown in the example.

\begin{minted}[fontsize=\footnotesize]{postgresql}
    -- Delete a table
    DROP TABLE [IF EXISTS] table_name 
    [CASCADE | RESTRICT];
\end{minted}

Here, the name of the DROP TABLE statement and a table name of your choosing is mandatory, the IF EXISTS clause is optional, and the CASCADE | RESTRICT clause is optional. The CASCADE | RESTRICT clause is used to specify what to do if the table has dependencies. If the table has dependencies, and CASCADE is specified, the table and all its dependencies will be deleted. If the table has dependencies and RESTRICT is specified, the table will not be deleted. The following example shows how to delete a table and its dependencies.

\begin{minted}[fontsize=\footnotesize]{postgresql}
    -- Delete a table and all of its dependencies
    DROP TABLE table_name CASCADE;
\end{minted}

Now, you can create databases and tables and alter and delete tables. You are now ready to start working with databases. The following section will guide you through inserting data into a table and how to query the data from the table.

\section{CRUD Operations in Tables}
CRUD is an acronym for Create, Read, Update, and Delete and is used to refer to the basic operations of working with data in a database. Creating databases and tables is a relatively simple task, but the real complexities of databases come with working with data in tables. This section will introduce you to the basic operations of inserting, reading, updating, and deleting data in tables. We start from the most straightforward operations and move to the more complex operations, where retrieval, interestingly enough, is the most difficult operation due to the sheer number of options available and the need to combine tables into coherent information.

\subsection{Inserting data in tables}
The following code snippet shows the available syntax for the insert operation. The syntax is the same for all database management systems, but options might differ. The options available in PostgreSQL are shown in the example below.

\begin{minted}[fontsize=\footnotesize]{postgresql}
    -- Syntax for inserting data into a table
    INSERT INTO table1(column1, column2, …)
    VALUES (value1, value2, …);
\end{minted}

Here, it is essential to understand that the name of the INSERT INTO statement and a table name of your choosing are mandatory, as well as the column names and the VALUES clause. The column names and the VALUES clause must match, and the order of the column names and the VALUES clause must match. Specifically, this means that in column1, value1 will be inserted. So, the person writing the SQL can define what comes first or if all table columns are filled with data. If a column of the table is not specified, the cell for that row inserted will be NULL. The following example shows how to insert data into a table.

\begin{minted}[fontsize=\footnotesize]{postgresql}
    -- Creating a table for the example
    CREATE TABLE cars (
        id SERIAL PRIMARY KEY,
        manufacturer VARCHAR(255) NOT NULL,
        owner VARCHAR(255) NOT NULL,
        last_update DATE
    );

    -- Inserting data into the table
    INSERT INTO cars(manufacturer, owner)
    VALUES ('Toyota', 'John Doe');

    -- Show inserted data
    SELECT * FROM cars;

    /* table output
    id | manufacturer | owner    | last_update
    1  | Toyota       | John Doe | NULL
    */
\end{minted}

The first part of the example creates a sample table. Here, we can see the ID as SERIAL, an integer that auto-increments. The manufacturer and owner are both VARCHAR(255) and must be filled. The last\_update is a DATE, and is not required to be filled. The second part of the example inserts data into the table. Here, we can see that the ID is not specified, and the last\_update is not specified. This means the ID will be auto-incremented, and the last\_update will be NULL. The third part of the example uses a new command further explored in a later section; right now, it suffices to know that a SELECT is about reading something from the database, while the star means all columns, and FROM defines from what table this should be read. So, the command shows shows the inserted data. The comment below shows the output of the select statement if run. Here, we can see that the ID is 1, the manufacturer is Toyota, the owner is John Doe, and the last\_update is NULL. One more option is available when inserting bulk data into a database, which the example below shows.

\begin{minted}[fontsize=\footnotesize]{postgresql}
    -- Inserting bulk data into the table
    INSERT INTO cars(manufacturer, owner)
    VALUES ('Toyota', 'John Doe'),
           ('Ford', 'Jane Doe'),
           ('Chevrolet', 'John Smith');

    -- Show inserted data
    SELECT * FROM cars;

    /* table output
    id | manufacturer | owner    | last_update
    1  | Toyota       | John Doe | NULL
    2  | Ford         | Jane Doe | NULL
    3  | Chevrolet    | John Smith | NULL
    */
\end{minted}

In this case, we save space by inserting multiple rows in one command. This type of mass insertion is good practice and should be used when inserting multiple rows, and it ensures we only specify the data structure once. The following section will guide you through updating data in a table.

\subsection{Deleting data in tables}
Deleting data in a table is straightforward, though it does not come without dangers. Everyone experienced with databases will have seen an inexperienced novice delete an entire table as the forgot a part of the statement. The following code snippet shows the available syntax for the delete operation. The syntax is the same for all database management systems, but options might differ. The options available in PostgreSQL are shown in the example below.

\begin{minted}[fontsize=\footnotesize]{postgresql}
    -- Syntax for deleting data from a table
    DELETE FROM table_name
    WHERE condition;
\end{minted}

Here, it is essential to understand that the name of the DELETE FROM statement and a table name of your choosing are mandatory; the WHERE clause is also compulsory. The WHERE clause is used to specify what rows to delete from the table. The following example shows how to delete data from a table.

\begin{minted}[fontsize=\footnotesize]{postgresql}
    -- Deleting data from the table
    DELETE FROM cars
    WHERE manufacturer = 'Toyota';

    -- Show inserted data
    SELECT * FROM cars;

    /* table output
    id | manufacturer | owner    | last_update
    2  | Ford         | Jane Doe | NULL
    3  | Chevrolet    | John Smith | NULL
    */
\end{minted}

In this example, we now deleted the first entry of the table. There are multiple dangers with the delete statement, and it is essential to be careful when using it. For example, yes, there was only one Toyota in the example above, but what if there were more? Then, all of them would be deleted. Therefore, it is vital to be as specific about what you want to delete, and the better option above would be to delete by ID, as the ID is unique, as this example shows.

\begin{minted}[fontsize=\footnotesize]{postgresql}
    -- Deleting data from the table
    DELETE FROM cars
    WHERE id = 2;

    -- Show inserted data
    SELECT * FROM cars;

    /* table output
    id | manufacturer | owner    | last_update
    3  | Chevrolet    | John Smith | NULL
    */
\end{minted}

In this example, we now deleted the second entry of the table. The last typical flaw implemented by most is simply forgetting the WHERE clause and deleting the entire table. This type of deletion is a common mistake, and it is essential to be careful when using the delete statement, as seen here.

\begin{minted}[fontsize=\footnotesize]{postgresql}
    -- Deleting data from the table
    DELETE FROM cars;

    -- Show inserted data
    SELECT * FROM cars;

    /* table output
    id | manufacturer | owner    | last_update
    */
\end{minted}

Remember, deleting data in a database is permanent, and a backup is the only way to recover the data. Databases have no recycle bin, and the data is gone forever. The following section will guide you through updating data in a table.

\subsection{Updating data in tables}
Updating data in a table is a straightforward operation used to change the data in a table. The following code snippet shows the available syntax for the update operation. The syntax is the same for all database management systems, but options might differ. The options available in PostgreSQL are shown in the example below.

\begin{minted}[fontsize=\footnotesize]{postgresql}
    -- Syntax for updating data in a table
    UPDATE table_name
    SET column1 = value1, column2 = value2, …
    WHERE condition;
\end{minted}

Here, it is essential to understand that the name of the UPDATE statement and a table name of your choosing is mandatory, the SET clause is compulsory, and the WHERE clause is required. The SET clause is used to specify what to update in the table, and the WHERE clause is used to identify what rows to update in the table. The following example shows how to update data in a table.

\begin{minted}[fontsize=\footnotesize]{postgresql}
    -- Updating data in the table
    UPDATE cars
    SET owner = 'Jane Doe'
    WHERE manufacturer = 'Ford';

    -- Show inserted data
    SELECT * FROM cars;

    /* table output
    id | manufacturer | owner    | last_update
    1  | Toyota       | John Doe | NULL
    2  | Ford         | Jane Doe | NULL
    3  | Chevrolet    | John Smith | NULL
    */
\end{minted}

Again, here, the best way to update a row is by ID, as the ID is unique, and the best way to be sure you are updating the correct row, but the example works, and the manufacturer Ford now has the owner Jane Doe. Just as with the delete statement, the update statement is permanent, and you have the same problems as with the DELETE statement if you forget where which will result in the following example.

\begin{minted}[fontsize=\footnotesize]{postgresql}
    -- Updating data in the table
    UPDATE cars
    SET owner = 'Jane Doe';

    -- Show inserted data
    SELECT * FROM cars;

    /* table output
    id | manufacturer | owner    | last_update
    1  | Toyota       | Jane Doe | NULL
    2  | Ford         | Jane Doe | NULL
    3  | Chevrolet    | Jane Doe | NULL
    */
\end{minted}

As you can see, all rows in the table now have the owner, Jane Doe. The following section will guide you through querying data from a table.

\subsection{Querying}
Querying data from a table is the most complex operation used to retrieve data from a table. The following code snippet shows the available syntax for the select operation. The syntax is the same for all database management systems, but options might differ. The options available in PostgreSQL are shown in the example below.

\begin{minted}[fontsize=\footnotesize]{postgresql}
    -- Syntax for querying data from a table
    SELECT column1, column2, …
    FROM table_name
    WHERE condition;
\end{minted}

Here, it is essential to understand that the name of the SELECT statement, the column names, and the FROM clause are mandatory, and the WHERE clause is also compulsory. The column names and the FROM clause must match, and the order of the column names and the FROM clause must match. The WHERE clause is used to specify what rows to retrieve from the table. The following example shows how to query data from a table.

\begin{minted}[fontsize=\footnotesize]{postgresql}
    -- Querying data from the table
    SELECT * FROM cars;

    /* table output
    id | manufacturer | owner    | last_update
    1  | Toyota       | John Doe | NULL
    2  | Ford         | Jane Doe | NULL
    3  | Chevrolet    | John Smith | NULL
    */
\end{minted}

In this example, we are retrieving all rows from the table. The star means all columns and the FROM defines from what table this should be read. The following example shows how to query specific columns from a table.

\begin{minted}[fontsize=\footnotesize]{postgresql}
    -- Querying specific columns from the table
    SELECT manufacturer, owner FROM cars;

    /* table output
    manufacturer | owner
    Toyota       | John Doe
    Ford         | Jane Doe
    Chevrolet    | John Smith
    */
\end{minted}

In this example, we retrieve the manufacturer and owner columns from the table. The following example shows how to query specific rows from a table.

\begin{minted}[fontsize=\footnotesize]{postgresql}
    -- Querying specific rows from the table
    SELECT * FROM cars
    WHERE manufacturer = 'Toyota';

    /* table output
    id | manufacturer | owner    | last_update
    1  | Toyota       | John Doe | NULL
    */
\end{minted}

In this example, we are retrieving all rows from the table where the manufacturer is Toyota. The following example shows how to query specific rows from a table and order the results. We will add a few more rows to the table for this example.

\begin{minted}[fontsize=\footnotesize]{postgresql}
    -- Inserting bulk data into the table
    INSERT INTO cars(manufacturer, owner)
    VALUES ('Toyota', 'John Doe'),
           ('Ford', 'Jane Doe'),
           ('Chevrolet', 'John Smith'),
           ('Toyota', 'Jane Smith'),
           ('Ford', 'John Smith'),
           ('Chevrolet', 'Jane Doe');

    -- Querying specific rows from the table, and order the results
    SELECT * FROM cars
    WHERE manufacturer = 'Toyota'
    ORDER BY owner;

    /* table output
    id | manufacturer | owner    | last_update
    1  | Toyota       | John Doe | NULL
    4  | Toyota       | Jane Smith | NULL
    */
\end{minted}

In this example, we are retrieving all rows from the table where the manufacturer is Toyota and ordering the results based on the owner. The following example shows how to query specific rows from a table and limit the results.

\begin{minted}[fontsize=\footnotesize]{postgresql}
    -- Querying specific rows from the table, and limiting the results
    SELECT * FROM cars
    WHERE manufacturer = 'Toyota'
    ORDER BY owner
    LIMIT 1;

    /* table output
    id | manufacturer | owner    | last_update
    1  | Toyota       | John Doe | NULL
    */
\end{minted}

In this example, we are retrieving all rows from the table where the manufacturer is Toyota, ordering the results by the owner, and limiting the results to 1. The following example shows how to query specific rows from a table and group the results.

\begin{minted}[fontsize=\footnotesize]{postgresql}
    -- Querying specific rows from the table, and grouping the results
    SELECT manufacturer, COUNT(*) FROM cars
    GROUP BY manufacturer;

    /* table output
    manufacturer | count
    Toyota       | 2
    Ford         | 2
    Chevrolet    | 2
    */
\end{minted}

In this example, we are retrieving the manufacturer and the count of rows from the table and grouping the results by the manufacturer. The following section expands on querying for data by allowing you to join multiple tables.

\subsection{Basic Joins}
For this section to make sense, as a reader, you first need to understand the different types of joins a database management system can perform. The different types of joins are shown in \cref{tab:1.postgresql-joins}.

\begin{table}[htb]
    \centering
    \resizebox{\textwidth}{!}{%
    \begin{tabular}{|l|l|}
    \hline
    \textbf{Join} & \textbf{Description} \\ \hline
    \textbf{INNER JOIN} & Returns records that have matching values in both tables \\ \hline
    \textbf{LEFT JOIN} & Returns all records from the left table, and the matched records from the right table \\ \hline
    \textbf{RIGHT JOIN} & Returns all records from the right table, and the matched records from the left table \\ \hline
    \textbf{FULL JOIN} & Returns all records when there is a match in either left or right table \\ \hline
    \end{tabular}%
    }
    \caption{Commonly used joins in PostgreSQL}
    \label{tab:1.postgresql-joins}
\end{table}

This section will show several examples, but they all work from the following code. The example is a simple example of a database with two tables designed to showcase the different types of joins. We are diverging from the cars example here, as it is not a good example to showcase the different types of joins. The example below shows how to create two tables and insert data into them.

\begin{minted}[fontsize=\footnotesize]{postgresql}
    -- Create table one
    CREATE TABLE table_one (
        num INTEGER PRIMARY KEY,
        name VARCHAR(255) NOT NULL
    );

    -- Create table two
    CREATE TABLE table_two (
        num INTEGER PRIMARY KEY,
        value VARCHAR(255) NOT NULL
    );

    -- Insert data into table one
    INSERT INTO table_one(num, name)
    VALUES (1, 'a'),
           (2, 'b'),
           (3, 'c');

    -- Insert data into table two
    INSERT INTO table_two(num, value)
    VALUES (1, 'xxx'),
           (2, 'yyy'),
           (5, 'zzz');
    
    -- Show inserted data
    SELECT * FROM table_one;
    SELECT * FROM table_two;

    /* table outputs
    num | name
    1   | a
    2   | b
    3   | c
    
    num | value
    1   | xxx
    2   | yyy
    5   | zzz
    */
\end{minted}

The example data is not a good way of creating a database, but an extreme example of how to exemplify the different join types, and should not be used for any other purpose.

\subsubsection{Inner Joins}
Joins allow us to take the content from two tables and combine them into one coherent result. To support the table above, show a few examples of how the different joins work. The following examples show how the different types of joins work.

\begin{figure}[H]
    \centering
    \includegraphics[width=0.5\textwidth]{content/1-relational-databases/figures/joins/innerjoin.png}
    \caption{An inner join; here only the overlapping orange area is returned, leaving out data not intersecting in both tables}
    \label{fig:1.innerjoin}
\end{figure}

As \cref{fig:1.innerjoin} shows, an inner join only returns the overlapping orange area and leaves out data not intersecting in both tables. There are implicit and explicit joins in SQL, and the example below shows how to perform an explicit inner join.

\begin{minted}[fontsize=\footnotesize]{postgresql}
    -- Querying specific rows from the table, and join 
    -- the results from another table
    SELECT table_one.num, table_one.name, table_two.value
    FROM table_one
    INNER JOIN table_two ON table_one.num = table_two.num;

    /* table output
    num | name | value
    1   | a    | xxx
    2   | b    | yyy
    */

    -- one can also make an implicit inner join
    -- the result is the same, but the syntax is different
    SELECT table_one.num, table_one.name, table_two.value
    FROM table_one, table_two
    WHERE table_one.num = table_two.num;
\end{minted}

The significance is that all rows with intersecting data in both tables are returned, and the rest is left out, leaving only two columns in the result. The following example shows how to perform a left join.

\subsubsection{Left and Right Joins}

\begin{figure}[H]
    \centering
    \includegraphics[width=0.5\textwidth]{content/1-relational-databases/figures/joins/leftjoin.png}
    \caption{A left join; here, the entire yellow area is returned, joined with the overlapping part from the B table}
    \label{fig:1.leftjoin}
\end{figure}

Here \cref{fig:1.leftjoin} shows a left join, and the entire yellow area is returned, joined with the overlapping part from the B table. The following example shows how to perform a left join.

\begin{minted}[fontsize=\footnotesize]{postgresql}
    -- Querying specific rows from the table, and join 
    -- the results from another table
    SELECT table_one.num, table_one.name, table_two.value
    FROM table_one
    LEFT JOIN table_two ON table_one.num = table_two.num;

    /* table output
    num | name | value
    1   | a    | xxx
    2   | b    | yyy
    3   | c    | NULL
    */
\end{minted}

Again, all the rows from the A side are kept here, while the B side is only kept if it intersects with the A-side. Notice the 3rd entry has a null value, as no data was found in the second table. The following example shows how to perform a right join.


\begin{figure}[H]
    \centering
    \includegraphics[width=0.5\textwidth]{content/1-relational-databases/figures/joins/rightjoin.png}
    \caption{A right join; here the entire yellow area is returned, joined with the overlapping part from the A table}
    \label{fig:1.rightjoin}
\end{figure}

Here \cref{fig:1.rightjoin} shows a right join, and the entire yellow area is returned, joined with the overlapping part from the A table. The same occurs here with the left join, but the roles are reversed. Let us see how a right join works.

\begin{minted}[fontsize=\footnotesize]{postgresql}
    -- Querying specific rows from the table, and join 
    -- the results from another table
    SELECT table_one.num, table_one.name, table_two.value
    FROM table_one
    RIGHT JOIN table_two ON table_one.num = table_two.num;

    /* table output
    num | name | value
    1   | a    | xxx
    2   | b    | yyy
    NULL| NULL | zzz
    */
\end{minted}

As predicted, the result is the same, but the roles are reversed. At this point, we see both NUM and NAME are null for the third entry, as no overlapping data was found in table one. The last example shows how to perform a full join.

\subsubsection{Full Joins and full outer joins}

\begin{figure}[H]
    \centering
    \includegraphics[width=0.5\textwidth]{content/1-relational-databases/figures/joins/fulljoin.png}
    \caption{A full join; everything and intersections are returned, but it also keeps nonintersecting data}
    \label{fig:1.fulljoin}
\end{figure}

Here \cref{fig:1.fulljoin} shows a full join, and everything and intersections are returned, but it also keeps nonintersecting data. The following example shows how to perform a full join.

\begin{minted}[fontsize=\footnotesize]{postgresql}
    -- Querying specific rows from the table, and join 
    -- the results from another table
    SELECT table_one.num, table_one.name, table_two.value
    FROM table_one
    FULL JOIN table_two ON table_one.num = table_two.num;

    /* table output
    num | name | value
    1   | a    | xxx
    2   | b    | yyy
    3   | c    | NULL
    NULL| NULL | zzz
    */
\end{minted}

Lastly, a full outer join can be performed, where in \cref{fig:1.fulljoin}, it would only take the A and the B side but leave out all intersections. Here is a code example of how to perform a full outer join.

\begin{minted}[fontsize=\footnotesize]{postgresql}
    -- Querying specific rows from the table, and join 
    -- the results from another table
    SELECT table_one.num, table_one.name, table_two.value
    FROM table_one
    FULL OUTER JOIN table_two ON table_one.num = table_two.num;

    /* table output
    num | name | value
    3   | c    | NULL
    NULL| NULL | zzz
    */
\end{minted}

\subsection{Practical joins}

The following example shows how to query specific rows from a table and join the results from another table. We will create the entire example from scratch so you can see how referential integrity works with the foreign key. In this example, users from the account table can own cars from the cars table.

\begin{minted}[fontsize=\footnotesize]{postgresql}
    -- Create account table
    CREATE TABLE account (
        id serial PRIMARY KEY,
        username VARCHAR (50) UNIQUE NOT NULL,
        created_on TIMESTAMP NOT NULL, 
        last_login TIMESTAMP
    );

    -- Create cars table
    CREATE TABLE cars (
        id SERIAL PRIMARY KEY,
        manufacturer VARCHAR(255) NOT NULL,
        owner INTEGER NOT NULL REFERENCES account (id),
        last_update DATE
    );

    -- Inserting data into the account table
    INSERT INTO account(username, created_on)
    VALUES ('John Doe', '2020-01-01'),
           ('Jane Doe', '2020-01-01'),
           ('John Smith', '2020-01-01'),
           ('Jane Smith', '2020-01-01');

    -- Inserting data into the cars table
    INSERT INTO cars(manufacturer, owner)
    VALUES ('Toyota', 1),
           ('Ford', 2),
           ('Chevrolet', 3),
           ('Toyota', 4),
           ('Ford', 3),
           ('Chevrolet', 2);

    -- Querying specific rows from the cars table, and join the 
    -- results from the account table
    SELECT cars.id, cars.manufacturer, account.username
    FROM cars
    JOIN account ON cars.owner = account.id;

    /* table output
    id | manufacturer | username
    1  | Toyota       | John Doe
    2  | Ford         | Jane Doe
    3  | Chevrolet    | John Smith
    4  | Toyota       | Jane Smith
    5  | Ford         | John Smith
    6  | Chevrolet    | Jane Doe
    */
\end{minted}

Hopefully, the above example puts the join examples into the usual context from earlier in the book.


\subsection{Nested Queries and Subqueries}
Nested queries, also called subqueries, allow you to use the result of one query as the input for another query. This combination is a powerful concept and will enable you to perform complex operations in a single query. The following example shows how to use a nested query.

\begin{minted}[fontsize=\footnotesize]{postgresql}
    -- Querying specific rows from the table, and use the 
    -- result as input for another query
    SELECT * FROM cars
    WHERE owner IN (SELECT id FROM account WHERE username = 'John Doe');

    /* table output
    id | manufacturer | owner    | last_update
    1  | Toyota       | 1        | NULL
    5  | Ford         | 3        | NULL
    */
\end{minted}

The above example shows how to use the result of the nested query as input for another query. The nested query is used to retrieve the ID of the user John Doe, and the result is used as input for the outer query. The outer query retrieves all cars owned by the user John Doe. Mind you, the above example would be better served with a join, but the example is used to show the concept of nested queries.
Why would one use a nested query? Nested queries allow for complex filtering criteria. You can use the results of one query to filter the results of another, enabling more granular control over data retrieval. This type of retrieval is particularly useful when applying multiple conditions or filters to your data. Also, nested queries can be used for data aggregation purposes, such as computing summary statistics or performing calculations on subsets of data. By nesting queries, you can calculate intermediate results before presenting the final result, providing a way to derive insights from your data. Also, a good use case for nested queries is correlating multiple datasets.

% ******************************* Chapter: ER, EER Modeling and Database Design ****************************
\chapter{ER, EER Modeling and Database Design}
\label{chap:relational:eer-modeling-and-database-design}
This chapter teaches basic ER and ER modeling and how to design a database from an ER model.

\section{The purpose of ER and EER modeling}
\section{Diagram Elements}
\section{ER versus EER modeling}
\section{Mapping to tables}

% ******************************* Chapter: Database Normalization ****************************
\chapter{Database Normalization}
\label{chap:relational:database-normalization}
This chapter teaches database normalization to the 4th normal form.

\section{What is database normalization?}
\section{Shorthand techniques}
\section{First normal form}
\section{Second normal form}
\section{Third normal form}
\section{Fourth normal form}
\section{Normalization of other formats}

% ******************************* Chapter: Advanced Relational Databases ****************************
\chapter{Advanced Relational Databases}
\label{chap:relational:advanced-relational-databases}
This chapter teaches advanced relational database concepts.

\section{Transactions}
\section{Indexes}
\section{Views}
\section{Stored Procedures}
\section{Triggers}
\section{User Defined Functions}
\section{Security}
\section{Performance Tuning}


